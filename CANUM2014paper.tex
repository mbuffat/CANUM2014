%% LyX 2.0.6 created this file.  For more info, see http://www.lyx.org/.
%% Do not edit unless you really know what you are doing.
\documentclass[10pt]{article}
\usepackage[latin9]{inputenc}
\usepackage{amsmath}
\usepackage{amssymb}
\begin{document}

\title{Webwork and SageCell 2 WEB 2.0 tools for learning scientific programming}
\maketitle
\begin{abstract}
Nowdays, numerical modeling, computer simulation and high performance
computing have become essential tools for research and innovation.
However the current training at the university does not adequately
prepare the students to master these new tools, as it has been shown
in a recent report of CSCI (O. Pironneau 2013). In particular, the
initial training of undergraduate students is inadequate (or non-existent)
in the field of mastering the scientific, mathematical and associated
programming tools. mastering of the scientific programming require
to write computer programms and to do math exercises. At the mechanical
engineering department of the University Claude Bernard Lyon 1, we
use for many years the online homework system Webwork {[}1{]} for
math and sciences courses and a Sage server {[}2{]} for learning scientific
programming.

With this experience, we start a project of a MOOC on scientific programming
with our colleagues in the mathematics department of Lyon 1, University
Joseph Fourier,Paris Sud and INRIA. The goal of this MOOC InProS is
to teach a methodology for scientific programming, based on problem
solving using Python and online exercises using SageCell and Webwork.
and tools, through a browser on PC or tablet. In the presentation
I will present these two Webwork and Sage tools, feedback from their
use, as well as the draft MOOC INPROS {[}3{]} \textquotedbl{}Introduction
� la programmation scientifique\textquotedbl{}.\end{abstract}
\begin{thebibliography}{1}
\bibitem{webwork} \textsc{WEBWORK}, \textsl{\textquotedbl{}syst�me
de devoirs de math�matique en ligne\textquotedbl{}: http://webwork.maa.org}
.

\bibitem{sage} \textsc{SAGE}, \textsl{\textquotedbl{}logiciel libre
de math�matiques en ligne\textquotedbl{}: http://www.sagemath.org}
.

\bibitem{inpros} \textsc{INPROS}, \textsl{\textquotedbl{}Introduction
� la Programmation Scientifique\textquotedbl{}: http://inpros.univ-lyon1.fr}
.\end{thebibliography}

\end{document}
